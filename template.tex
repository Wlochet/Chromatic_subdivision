\documentclass{endm}
\usepackage{endmmacro}
\usepackage{graphicx}

% The following is enclosed to allow easy detection of differences in
% ascii coding.
% Upper-case    A B C D E F G H I J K L M N O P Q R S T U V W X Y Z
% Lower-case    a b c d e f g h i j k l m n o p q r s t u v w x y z
% Digits        0 1 2 3 4 5 6 7 8 9
% Exclamation   !           Double quote "          Hash (number) #
% Dollar        $           Percent      %          Ampersand     &
% Acute accent  '           Left paren   (          Right paren   )
% Asterisk      *           Plus         +          Comma         ,
% Minus         -           Point        .          Solidus       /
% Colon         :           Semicolon    ;          Less than     <
% Equals        =           Greater than >          Question mark ?
% At            @           Left bracket [          Backslash     \
% Right bracket ]           Circumflex   ^          Underscore    _
% Grave accent  `           Left brace   {          Vertical bar  |
% Right brace   }           Tilde        ~

\newcommand{\Nat}{{\mathbb N}}
\newcommand{\Real}{{\mathbb R}}
\def\lastname{Please list your Lastname here}

\begin{document}

% DO NOT REMOVE: Creates space for Elsevier logo, ScienceDirect logo
% and ENDM logo
\begin{verbatim}\end{verbatim}\vspace{2.5cm}

\begin{frontmatter}

\title{Bispindle in strongly connected digraphs with large chromatic number.}

\author{Nathann Cohen\thanksref{mail1}}
\address{CNRS, LRI, Univ. Paris Sud, Orsay, France}

\author{Fr\'ed\'eric Havet\thanksref{mail2}}
\address{ Univ. C\^ote d'Azur, CNRS, I3S, INRIA, France}


\author{William Lochet\thanksref{mail3}}
\address{ Univ. C\^ote d'Azur, CNRS, I3S, INRIA, and LIP, ENS Lyon, France}

\author{Raul Lopes\thanksref{mail4}}
\address{Departamento de Computa\c{c}ao, Universidade Federal do Cear\'a, Fortaleza, Brazil}



   \thanks[mail1]{\texttt{\normalshape
  nathann.cohen@gmail.com}} 
  \thanks[mail2]{\texttt{\normalshape
  frederic.havet@cnrs.fr}} 
  \thanks[mail3]{\texttt{\normalshape
  william.lochet@gmail.com}}
  \thanks[mail4]{\texttt{\normalshape
  raul.wayne@gmail.com}}


\begin{abstract}
A {\it $(k_1+k_2)$-bispindle} is the union of $k_1$  $(x,y)$-dipaths and $k_2$ $(y,x)$-dipaths, all these dipaths being pairwise internally disjoint.
Recently, Cohen et al. showed that for every $(1,1)$- bispindle $B$, there exists an integer $k$ such that every strongly connected
digraph with chromatic number greater than $k$ contains a subdivision of $B$. We investigate generalisations of
this result by first showing constructions of strongly connected digraphs with large chromatic number without any $(3,0)$-bispindle
or $(2,2)$-bispindle. Then we show that strongly connected digraphs with large chromatic number contains a 
$(2,1)$-bispindle, where at least one of the $(x,y)$-dipaths and the $(y,x)$-dipath are long.    
\end{abstract}

\begin{keyword}
Digraph, chromatic number, subdivision.
\end{keyword}

\end{frontmatter}



\section{Introduction}
Throughout this paper, the {\it chromatic number} of a digraph $D$, denoted by $\chi(D)$, is the chromatic number of its underlying graph. 
In a digraph $D$, a \emph{directed path}, or \emph{dipath} is an oriented path where all the arcs are oriented form the initial vertex towards the terminal vertex.
A {\it $k$-spindle} is the union of $k$ internally disjoint $(x,y)$-dipaths for some vertices $x$ and $y$. Vertex $a$ is said to be the {\it tail} of the spindle and $b$ its {\it head}.
A {\it $(k_1+k_2)$-bispindle} is the internally disjoint union of a $k_1$-spindle
with tail $x$ and head $y$ and a  $k_2$-spindle with tail $y$ and head $x$.
In other words, it is the union of $k_1$  $(x,y)$-dipaths and $k_2$ $(y,x)$-dipaths, all of these dipaths being pairwise internally disjoint.

A classical result due to Gallai, Hasse, Roy and Vitaver is the following. 
\begin{theorem}[Gallai~\cite{Gal68}, Hasse~\cite{Has64}, Roy~\cite{Roy67}, Vitaver~\cite{Vit62}]\label{thm:gallairoy}
	If $\chi(D) \geq k$, then $D$ contains a directed path of length $k-1$.  
\end{theorem}   

This raises the question of which digraphs are subdigraphs of all digraphs with large chromatic number.



A classical theorem by Erd\H{o}s~\cite{Erd59} implies that if $H$ is a digraph containing a cycle, there exist digraphs with arbitrarily high 
chromatic number with no subdigraph isomorphic to $H$. Thus the only possible candidates to generalise Theorem \ref{thm:gallairoy} are the {\it oriented trees} that are orientations of trees.
Burr\cite{Burr80} proved that every $(k-1)^2$-chromatic digraph contains every oriented tree of order $k$ 
and conjectured an upper bound of $2k-2$.
The best known upper bound, due to Addario-Berry et al.~\cite{AHS+13}, is in $(k/2)^2$.
%However, for paths with two blocks ({\it blocks} are maximal directed subpaths), the best possible upper bound is known.

%\begin{theorem}[Addario-Berry et al.~\cite{AHT07}]\label{thm:2blocks}
%Let $P$ be an oriented path with two blocks on $n > 3$ vertices, then every digraph with chromatic number (at least) $n$ contains $P$. 
%\end{theorem}
However the following celebrated theorem of Bondy shows that  the story does not stop there.

\begin{theorem}[Bondy~\cite{Bon76}]\label{thm:bondy}
Every strongly connected digraph with chromatic number at least $k$ contains a directed cycle of length at least $k$.
\end{theorem} 

The strong connectivity assumption is indeed necessary, as transitive tournaments contain no directed cycle but can have arbitrarily  high chromatic number. 

Observe that a directed cycle of length at least $k$ can be seen as a subdivision of $\vec{C}_k$, the directed cycle of length $k$.
Recall that a {\it subdivision} of a digraph $F$ is a digraph that can be obtained from $F$ by replacing each arc $uv$ by a directed path from $u$ to $v$.

\begin{conjecture}[Cohen et al.  \cite{CHLN16}]\label{conj:cycle-sub}
For every cycle $C$, there exists a constant $f(C)$ such that every strongly connected digraph with chromatic number at least $f(C)$ contains a subdivision of $C$.
\end{conjecture}

The strong connectivity assumption is also necessary in Conjecture~\ref{conj:cycle-sub} as shown by Cohen et al. in  \cite{CHLN16}.
%where they proved the existence of digraphs with large chromatic number such that every cycle has high number of blocks.
In the same paper, Conjecture~\ref{conj:cycle-sub} was confirmed for cycles with two blocks (i.e. maximal directed subpaths of the cycle) and the antidirected cycle of length $4$.
More precisely, denoting by $C(k,\ell)$ the cycle on two blocks, one of length $k$ and the other of length $\ell$,
 Cohen et al. \cite{CHLN16} proved the following.
\begin{theorem}\label{th:ckl}
Every strongly connected digraph with chromatic number at least $O((k+\ell)^4)$ contains a subdivision of $C(k,\ell)$.
\end{theorem} 
The bound has recently been improved to $O((k+\ell)^2)$  by Kim et al. \cite{KKPM}.  


A subdivision of $C(k,\ell)$ can be seen as $2$-spindle made of  two internally disjoint dipaths, one of length at least $k$ and one of length at least $\ell$.
% and a subdivision of $\vec{C}_{k+\ell}$ can be seen as two internally disjoint dipaths in opposite direction between two vertices, one length at least $k$ and one of length at least $\ell$.
In this paper, we generalize this and study the existence of subdivision of spindles and bispindles in strongly connected digraphs with large chromatic number.
Our first result is to give constructions for the following theorem: 
\begin{theorem}
For every integer $k$, there exists a strongly connected digraph $D$ with $\chi(D) >k$ that contains no $3$-spindle and no $(2+2)$-bispindle. 
\end{theorem}
Therefore, the most we can expect in all strongly connected digraphs with large chromatic number are $(2+1)$-bispindle. Let $B(k_1,k_2;k_3)$ denote the $(2+1)$-bispindle formed by three internally disjoint paths between two vertices $x,y$,
two $(x,y)$-directed paths, one of size at least $k_1$, the other of size at least $k_2$, and one $(y,x)$-directed path of size at least $k_3$.
We conjecture the following.

\begin{conjecture}\label{conj:3chemins}
There is a function $g:\mathbb{N}^3 \rightarrow \mathbb{N}$ such that  every strongly connected digraph with chromatic number at least $g(k_1,k_2,k_3)$ contains
a subdivision of $B(k_1,k_2;k_3)$.
\end{conjecture}

As an evidence, we prove the following theorem:

\begin{theorem}\label{thm:Pk1k}
For every positive integer $k$, there is a constant $\gamma_k$ such that every strongly connected digraph witch chromatic number greater than $\gamma_k$ contains
a subdivsion of $B(k,1;k)$.
\end{theorem}

The value of $\gamma_k$ is the above theorem is huge, and certainly not best possible. We get a better bound for subdivision of $B(k,1;k)$.
\begin{theorem}\label{th:P11k}
Let $k \geq 3$ be an integer and let $D$ be a strong digraph. If $\chi(D) >  (2k-2)(2k-3)$, then $D$ contains a subdivision of $B(k,1;1)$.
\end{theorem}


\section{Proof of Theorem~\ref{thm:Pk1k}}
%%%%%%%%%%%%%%%%%%%%%%%%%%

We prove Theorem~\ref{thm:Pk1k} by the contrapositive.
We consider a  digraph $D$ without any subdivision of $B(k,1;k)$. We shall prove that $\chi(D)\leq \gamma_k$.


The general idea is to use the following easy lemma.
\begin{lemma}\label{lem:contrac}
Let $D$ be a digraph, $D_1 \dots D_l$ be disjoint subdigraphs of $D$ and $D'$ the digraph obtained by contracting each $D_i$ into
one vertex $d_i$. Then $\chi(D) \leq \chi(D')\cdot \max\{\chi(D_i) \mid 1\leq i \leq l\}$.
\end{lemma}

The key is to find appropriate subdigraphs $D_i$.
To do so, we consider some particular collections of directed cycles~:
a collection ${\cal C}$ of directed cycles is {\it $k$-suitable} if 
all cycles of ${\cal C}$ have length at least $8k$, and
any two distinct cycles $C_i,C_j\in\mathcal C$ intersect on  a subpath of order at most $k$.
A \textit{component} of $\mathcal{C}$ is a connected component of the underlying graph of the digraph $\bigcup {\cal C}$ which is the union of
cycles of $\mathcal{C}$.		

Consider $\mathcal{C}$ be a maximal $k$-suitable collection of cycles in $D$. Let $D'$ be the digraph obtained by contracting 
every strong component $S$ of $\bigcup \mathcal{C}$ (which is $\bigcup {\cal S}$ for some component ${\cal S}$ of ${\cal C}$) into one vertex. For each connected component ${\cal S}_i$ we call $s_i$ the
new vertex created. To apply Lemma~\ref{lem:contrac}, we shall prove in the next two lemmas that for every component $\mathcal{S}$ of $\mathcal{C}$, the digraph $D[{\cal S}]$ induced by $D$ on the vertices
of $\bigcup {\cal S}$ has bounded chromatic number and that $\chi(D')\leq 8k$.

\begin{lemma}\label{lem:DC}
Let ${\cal C}$ be a $k$-suitable collection of directed cycles in a $B(k,1;k)$-free digraph $D$.
There exists a constant $\beta_k$ such that, for every component $\mathcal{S}$ of $\mathcal{C}$, 
we have $\chi(D[\mathcal{S}]) \leq \beta_k$.
\end{lemma} 
\noindent {\bf Sketch of proof}:
We first consider $\bigcup{\cal S}$ which is a subdigraph of $D[{\cal S}]$.  We prove by induction on the number of cycles in ${\cal S}$ that this digraph admits a proper colouring $\phi$ with $\alpha_k=2\cdot (6k^2)^{3k} + 14k$ colours satisfying the following additional property, called {\it rainbow property} : the vertices of each subpath of length at most $7k$ of each cycle of $\mathcal{S}$ get different colours.

We then define a sort of Breadth-First-Search for $\mathcal{S}$.
Let $C_0$ be a cycle of $\mathcal{S}$ and set $L_0 = \{C_0\}$. We build  the levels $L_i$ inductively
until all cycles of $\mathcal{S}$ are put in a level : $L_{i+1}$ consists of every cycle $C_l$ not in $\bigcup_{j \leq i} L_j$ such that 
there exists a cycle in $L_i$ intersecting $C_l$.  For every $C_l\in L_{i+1}$, we choose one of the cycles $L_{i}$ intersecting it to be its {\it  father}. 
%Henceforth every cycle in $L_{i+1}$ has a unique  father even though it might intersect many cycles of $L_i$.
%A cycle $C$ is an {\it ancestor} of $C'$ if there is a sequence $C=C_1, \dots , C_q=C'$ such that $C_i$ is the father of $C_{i+1}$ for all $i\in [q-1]$.
For a vertex $x$ of $\bigcup \mathcal{S}$, we say that $x$ \textit{belongs to} level $L_i$ if
$i$ is the smallest integer such that there exists a cycle in $L_i$ containing $x$. 
%Observe that the vertices of each cycle $C_l$ of ${\cal S}$ belong to consecutive levels, that is there exists $i$ such that $V(C_l)\subseteq L_i\cup L_{i+1}$.

We partition the arc set of of $D[{\cal S}]$  in $(A_0,A_1, A_2)$, where
\begin{itemize}
	\item $A_0$ is the set of arcs of $D[{\cal S}]$ which ends belong to the same level, and
	\item $A_1$ is the set of arcs of $D[{\cal S}]$ which ends belong to different levels $i$ and $j$ with $ | i - j| < k$.	
	\item $A_2$ is the set of arcs of $D[{\cal S}]$ which ends belong to different levels $i$ and $j$ with $ | i - j| \geq k$.
\end{itemize} 

For $i\in [3]$, let $D_i$ be the spanning subdigraph of $D[{\cal S}]$ with arc set $A_i$. It is well-known that $\chi(D[{\cal S}])\leq \chi(D_1)\times \chi(D_2) \times \chi(D_3)$.

Clearly, $\chi(D_1)\leq k$, and we show that $\chi(D_2)\leq 4k^2 +2$. To bound $\chi(D_0)$ we partition the vertex set according to the above-mention colouring $\phi$ of $\bigcup {\cal S}$.
Using the rainbow property, we prove that the subdigraph of $D_0$ induced by the vertices of colour $c$ has chromatic number at most $2\cdot (4k)^{4k} + 1$ for all colour $c$. Hence $\chi(D_0) \leq (2\cdot (4k)^{4k} + 1) \alpha_k$.
This gives the result for $\beta_k= k (4k^2+2) (2\cdot (4k)^{4k} + 1) \alpha_k$.
\hfill $\square$



\begin{lemma}\label{cl:nocycle}
$\chi(D') \leq 8k$.
\end{lemma} 
\noindent {\bf Proof.}
First note that since $D$ is strongly connected so is $D'$. 

Suppose for a contradiction that $\chi(D') > 8k$. By Theorem \ref{thm:bondy}, there exists
a directed cycle $C = (x_1, x_2, \dots , x_l, x_1)$ of length at least $8k$. 	
For each vertex $x_j$ that corresponds to a $s_i$ in $D$, the arc $x_{j-1}x_j$ corresponds in $D$ to an arc whose head is a vertex $p_i$ of $S_i$ and the arc $x_jx_{j+1}$ corresponds to an arc whose tail is a vertex $l_i$ of $S_i$. Let $P_j$ be the dipath 
		from $p_i$ to $l_i$ in $\bigcup \mathcal{C}$. Note that this path intersects the elements of $S_i$ only along a subdipath. 
Let $C'$ be the cycle obtained from $C$ where we replace all contracted vertices $x_j$ by the path $P_j$. 
First note that $C'$ has length at least $8k$. Moreover, a cycle of $\mathcal{C}$  can
intersect $C'$ only along one $P_j$, because they all correspond to different strong components of $\bigcup \mathcal{C}$. Thus $C'$ intersects  each cycle of $\mathcal{C}$ on a subdipath. Moreover this subdipath has length smaller than $k$ for otherwise $D$ would contain 
a subdivision of $B(k,1;k)$. So $C'$ is a directed cycle of length at least $8k$ which intersects every cycle of $\mathcal{C}$ along a 
subdipath of length less than $k$. This contradicts the maximality of $\mathcal{C}$.
\hfill $\square$

\medskip

Using Lemma \ref{lem:contrac} with
Claim~\ref{cl:nocycle} and Lemma~\ref{lem:DC}, we get that $\chi(D) \leq 8k \cdot \beta_k$.
This proves Theorem~\ref{thm:Pk1k} for $\gamma_k=8k\cdot \beta_k$.



\begin{thebibliography}{10}\label{bibliography}


%\bibitem{AHT07}
%L.~Addario-Berry, F.~Havet, and S.~Thomass{\'e}.
%\newblock Paths with two blocks in $n$-chromatic digraphs.
%\newblock \emph{Journal of Combinatorial Theory, Series B}, 97
 % (4): 620--626, 2007.

\bibitem{AHS+13}
L.~Addario-Berry, F.~Havet, C.~L. Sales, B.~A. Reed, and S.~Thomass{\'e}.
\newblock Oriented trees in digraphs.
\newblock \emph{Discrete Mathematics}, 313 (8): 967--974,
  2013.

%\bibitem{AC+16}
%P. Aboulker, N. Cohen, F. Havet, W. Lochet, P. Moura and S. Thomassé
%\newblock Subdivisions in digraphs of large out-degree or large dichromatic number
%\newblock arXiv:1610.00876

 %\bibitem{BeWo83} A. Benhocine and A. P. Wojda.
 %\newblock On the existence of specified cycles in a tournament.
 %\newblock \emph{J. Graph Theory}, 7: 469-473, 1983.

   \bibitem{Bon76} J. A. Bondy, Disconnected orientations and a conjecture of Las Vergnas,
{\it J. London Math. Soc. (2)}, {\bf 14} (2) (1976), 277--282.

%
%  \bibitem{BoMu08}
%J.A. Bondy and U.S.R. Murty.
%\newblock {\em {G}raph {T}heory}, volume 244 of {\em Graduate Texts in
%  Mathematics}.
%\newblock Springer, 2008.


%\bibitem{Bro41}
%R.~L. Brooks.
%\newblock On colouring the nodes of a network.
%\newblock {\em Proc. Cambridge Philos. Soc.}, 37:194--197, 1941.

\bibitem{Burr80}
S.~A. Burr.
\newblock Subtrees of directed graphs and hypergraphs.
\newblock In \emph{Proceedings of the 11th Southeastern Conference on
  Combinatorics, Graph theory and Computing}, pages 227--239, Boca Raton - FL,
  1980. Florida Atlantic University.

\bibitem{Bur82} S. A. Burr,
Antidirected subtrees of directed graphs.
{\it Canad. Math. Bull.} {\bf 25} (1982), no. 1, 119--120.

\bibitem{CHLN16}
N. Cohen, F. Havet, W. Lochet, and N. Nisse.
\newblock Subdivisions of oriented cycles in digraphs with large chromatic number.
\newblock arXiv:1605.07762

\bibitem{Erd59}
P.~Erd{\H{o}}s.
\newblock Graph theory and probability.
\newblock {\em Canad. J. Math.}, 11:34--38, 1959.

\bibitem{ErHa66}
P.~Erd{\H{o}}s and A. Hajnal.
\newblock On chromatic number of graphs and set-systems.
\newblock {\em Acta Mathematica Academiae Scientiarum Hungarica}, 17(1-2):61--99, 1966.



\bibitem{Gal68}
T.~Gallai.
\newblock On directed paths and circuits.
\newblock In \emph{Theory of Graphs (Proc. Colloq. Titany, 1966)}, pages
  115--118. Academic Press, New York, 1968.

%\bibitem{Gya92}
%A. Gy\'arf\'as.
%\newblock Graphs with $k$ odd cycle lengths.
%\newblock {\it Discrete Math.}, 103, pp. 41--48, 1992.


\bibitem{Has64}
M.~Hasse.
\newblock Zur algebraischen bergr\"und der graphentheorie {I}.
\newblock \emph{Math. Nachr.}, 28: 275--290, 1964.


\bibitem{KKPM}
R. Kim, SJ. Kim, J. Ma; B. Park
\newblock Cycles with two blocks in $k$-chromatic digraphs
\newblock arXiv:1610.05839



\bibitem{Roy67}
B.~Roy.
\newblock Nombre chromatique et plus longs chemins d'un graphe.
\newblock \emph{Rev. Francaise Informat. Recherche Op\'erationnelle},
  1 (5): 129--132, 1967.


%\bibitem{Sum81}
%D.~P. Sumner.
%\newblock Subtrees of a graph and the chromatic number.
%\newblock In {\em The theory and applications of graphs (Kalamazoo, Mich.,
%  1980)}, pages 557--576. Wiley, New York, 1981.

\bibitem{Vit62}
L.~M. Vitaver.
\newblock Determination of minimal coloring of vertices of a graph by means of
  boolean powers of the incidence matrix.
\newblock \emph{Doklady Akademii Nauk SSSR}, 147: 758--759, 1962.




\end{thebibliography}

\end{document}