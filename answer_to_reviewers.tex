\documentclass[]{article}

\usepackage{fullpage}
\usepackage{color}
\usepackage{amsmath}
\newcommand{\joanna}[1]{\textcolor{blue}{(JOANNA: #1)}}

\newcommand{\fred}[1]{{\color{red} [Fred: #1]}}

\newcounter{cptcomments}
\setcounter{cptcomments}{1}

\newenvironment{comment}
{ \noindent\textbf{Comment~\arabic{cptcomments}:~}\ignorespaces }
{ \addtocounter{cptcomments}{1}\hfill\medskip\par\noindent\ignorespacesafterend }

\newenvironment{answer}
{ \noindent\textbf{Answer:~}\ignorespaces }
{ \hfill\bigskip\par\noindent\ignorespacesafterend }

\newenvironment{sketchofproof}{\noindent\emph{Proof sketch. }\ignorespaces}{\hfill\qed\medskip\par\noindent\ignorespacesafterend}

\begin{document}


\noindent\begin{tabular}{ll}
\textbf{Manuscript Title:} & Bispindles in strongly connected digraphs with large chromatic number \\[1mm]
\textbf{Authors:} & N. Cohen, F. Havet, W. Lochet, and R. Lopes\\[1mm]
Submitted for publication to:& \emph{Algorithmica}
\end{tabular}

\bigskip
\begin{center}
  {\Large{\textbf{Response to reviewers}}}
\end{center}
\bigskip


First of all, we would like to thank the anonymous reviewers for their time and
efforts, and for providing valuable feedback in their reports. We carefully
considered all the comments and suggested revisions. 


\medskip

We corrected all spotted misprints and unified the notations.

\medskip

We also reorganized the paper as suggested by one of the referees.

Firstly, we included the easy results (Theorem 9 and Proposition 10) into the introduction to motivate possible readers and attract them to the problem.

Secondly, we reorganized the proofs in Subsection 4.1. We moved Lemma 25 at the beginning of this subsection so that the reader knows the reason why we prove Lemmas 26-30.


\medskip 


%\bibliographystyle{plain}
%\bibliography{references}



\end{document}
