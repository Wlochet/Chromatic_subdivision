% EJC papers *must* begin with the following two lines.
\documentclass[utf8,11pt]{article}
\usepackage[utf8]{inputenc}
\usepackage{fullpage}


% Please remove all other commands that change parameters such as
% margins or pagesizes.

% only use standard LaTeX packages
% only include packages that you actually need

% we recommend these ams packages
\usepackage{amsthm,amsmath,amssymb}
\usepackage{algorithm2e}
\newcommand{\qedclaim}{\hfill $\diamond$ \medskip}
\newenvironment{proofclaim}{\noindent{\em Proof of the claim.}}{\qedclaim}

% we recommend the graphicx package for importing figures
%\usepackage{graphicx}

% use this command to create hyperlinks (optional and recommended)
\usepackage[colorlinks=true,citecolor=black,linkcolor=black,urlcolor=blue]{hyperref}

% use these commands for typesetting doi and arXiv references in the bibliography
\newcommand{\doi}[1]{\href{http://dx.doi.org/#1}{\texttt{doi:#1}}}
\newcommand{\arxiv}[1]{\href{http://arxiv.org/abs/#1}{\texttt{arXiv:#1}}}

% all overfull boxes must be fixed;
% i.e. there must be no text protruding into the margins


% declare theorem-like environments
\theoremstyle{plain}
\newtheorem{theorem}{Theorem}
\newtheorem{lemma}[theorem]{Lemma}
\newtheorem{corollary}[theorem]{Corollary}
\newtheorem{proposition}[theorem]{Proposition}
\newtheorem{fact}[theorem]{Fact}
\newtheorem{observation}[theorem]{Observation}
\newtheorem{claim}{Claim}[theorem]

\theoremstyle{definition}
\newtheorem{definition}[theorem]{Definition}
\newtheorem{example}[theorem]{Example}
\newtheorem{conjecture}[theorem]{Conjecture}
\newtheorem{open}[theorem]{Open Problem}
\newtheorem{problem}[theorem]{Problem}
\newtheorem{question}[theorem]{Question}

\theoremstyle{remark}
\newtheorem{remark}[theorem]{Remark}
\newtheorem{note}[theorem]{Note}

\usepackage{thm-restate}

\newcommand{\william}[1]{{\color{red}{\bf William:} #1}}
\newcommand{\nathann}[1]{{\color{blue}{\bf Nathann:} #1}}

\newenvironment{subproof}{\par\noindent {\it Subproof}.\ }{\hfill$\lozenge$\par\vspace{11pt}}

\newcommand{\cste}{ 12k \cdot (2\cdot \col + 18k) \cdot (2 \cdot \dr)}
\newcommand{\qj}{\bf cste2}
\newcommand{\dr}{ k^2}
\newcommand{\col}{(8k^2)^{4k}}
\newcommand{\free}{subdivision-free}

\DeclareMathOperator{\Dis}{Dis}


%%%%%%%%%%%%%%%%%%%%%%%%%%%%%%%%%%%%%%%%%%%%%%%%%%%%%%%

% if needed include a line break (\\) at an appropriate place in the title

\title{Some paths and chromatic number}



\begin{document}

\maketitle

\section{Introduction}
Throughout this paper, the chromatic number of a digraph $D$ is the chromatic number of the underlying graph. 

A classical result due to Gallai, Hasse, Roy and Vitaver says that the chromatic number of a graph $G$ equals one plus the length of a longest path in an orientation 
of $G$ chosen to minimise this path's length. This surprising characterisation of the chromatic number with the orientation of a graph motivates
the following question: what are the subdigraphs of digraphs with large chromatic number.
In this setting, which will be the one that interests us from now on, the Theorem can be stated as follows: 

\begin{theorem}[Gallai~\cite{Gal68}, Hasse~\cite{Has64}, Roy~\cite{Roy67}, Vitaver~\cite{Vit62}]\label{thm:gallairoy}
	If $\chi(D) \geq k$ then $D$ contains a dipath of length $k+1$.  
\end{theorem}   

A famous theorem by Erd\H{o}s~\cite{Erd59} states that there exists graph with arbitrarily high girth and
chromatic number. This means that if $H$ is a digraph containing a cycle, there exists digraphs with arbitrarily high 
chromatic number without any copy of $H$. Thus the only possible candidates to generalise Theorem \ref{thm:gallairoy} are oriented
of trees.
Burr proved that every $(k-1)^2$-chromatic digraph contains every oriented tree of order $k$ and conjectured the following:
\begin{conjecture}[Burr~\cite{Burr80}]\label{cnj:tn}
	  Every $(2k-2)$-chromatic digraph $D$ contains a copy of any oriented tree $T$ of order $k$.
\end{conjecture}
The best known upper bound, due to Addario-Berry et al.~\cite{AHS+13}, is in $(k/2)^2$.
However, for paths with two blocks ({\it blocks} are maximal directed subpaths), the best possible upper bound is known:

\begin{theorem}[Addario-Berry et al.~\cite{AHT07}]\label{thm:2blocks}
Let $P$ be an oriented path with two blocks on $n$ vertices.
\begin{itemize}
\item If $n=3$, then every digraph with chromatic number greater than $n$ contains $P$
\item If $n\geq 4$, then every digraph with chromatic number greater than $n-1$ contains $P$
\end{itemize}
\end{theorem}

Another interesting generalisation is to look for subdivision instead of subdigraph. 
In \cite{CHLN16} Cohen et al. proved the following:

\begin{theorem}\label{dkb}
For any positive integer $b$ and $k$, there exists an acyclic digraph $D_{k,b}$ such that any cycle in $D_{k,b}$ has at least $b$ blocks and $\chi(D_{k,b}) > k$.
\end{theorem}

This means that even in the case of subdivision, one can only hope to force oriented trees. 
However, if we require the digraph to be strongly connected, the picture is a bit different. 
The following famous theorem by Bondy can be seen as a generalisation of the Gallai-Roy-Hasse-Vitaver's result:

\begin{theorem}[Bondy~\cite{Bon76}]\label{thm:bondy}
Every strong digraph of chromatic number at least $k$ contains a directed cycle of length at least $k$.
\end{theorem} 

Let $C(k,l)$ denote the cycle on two blocks, one of length at least $k$ and the other of length at least $l$.
In \cite{CHLN16}, Cohen et al. proved the following result:
\begin{theorem}\label{th:ckl}
Every strongly connected digraph of chromatic number greater than $O((k+l)^4)$ contains a subdivision of $C(k,l)$.
\end{theorem} 
The bound has recently been improved by Kim et al. in \cite{KKPM} to $O((k+l)^2)$.  

A subdivision of $C(k,l)$ can be seen as two disjoint dipaths $P_l$ of length at least $l$ and $P_k$ of length at least $k$,
between the same pair of vertices. The goal of this paper is to try to generalise this results to more than two paths. 
Let $P(k_1,k_2;k_3)$ be the digraph formed by three internally disjoint paths between two vertices $x,y$, two $(x,y)$-dipaths,
one of size at least $k_1$, the other of size at least $k_2$, and one $(y,x)$-dipath of size at least $k_3$. 
First we give a construction of digraphs with arbitrarily large chromatic number such that for any pair of vertices
$x$ and $y$, there is no three disjoint dipaths from $x$ to $y$ or four disjoint dipath, two from $x$ to $y$ and two form $y$ to $x$. 
This implies that $P(k_1,k_2;k_3)$ is the only possible candidate if we want to generalise Theorem \ref{th:ckl}. 
Then we present a proof for forcing $P(k,1;1)$: 

\begin{theorem}\label{th:P11k}
Let $D$ be a strongly connected digraph. If $\chi(D) >  (2k-1)\cdot (2k)$, then $D$ contains a subdivision of $P(k,1;1)$
\end{theorem}

and finally, building on the proof of Theorem \ref{th:P11k}, we present our main result: 

\begin{theorem}\label{th:main}
Let $D$ be a strongly connected digraph. If $\chi(D) > \cste$, then $D$ contains a subdvision of $P(k,1;k)$.
\end{theorem}



\section{Definitions and preliminaries}
%%%%%%%%%%%%%%%%%%


We denote by $[k]$ the set of integers $\{1, \dots , k\}$.

Let $F$ be a digraph.
A digraph $D$ is said to be {\it $F$-subdivision-free}, if it contains no subdivision of $F$.

The {\it union} of two digraphs $D_1$ and $D_2$ is the digraph  $D_1\cup D_2$ defined by $V(D_1\cup D_2) = V(D_1)\cup V(D_2)$ and 
$A(D_1\cup D_2) = A(D_1)\cup A(D_2)$.
If ${\cal D}$ is a set of digraphs, we denote by $\bigcup {\cal D}$ the union of the digraphs, i.e. $V(\bigcup {\cal D}) =\bigcup_{D\in {\cal D}} V(D)$
and $A(\bigcup {\cal D}) =\bigcup_{D\in {\cal D}} A(D)$.

\begin{lemma}\label{lem:decomp}
Let $D_1$ and $D_2$ be two digraphs.
$\chi(D_1\cup D_2) \leq \chi(D_1)\times \chi(D_2)$.
\end{lemma}



{\bf def de dipath}

Let $P$ be a path. We denote by $s(P)$ its initial vertex and by $t(P)$ its terminal vertex.
The subpath of $P$ with initial vertex $u$ and terminal vertex $v$ is denoted by $P[u,v]$.
Similarly, if $C$ is a directed cycle, we denote by $C[u,v]$ the subdipath of $C$ with initial vertex $u$ and terminal vertex $v$.


Let $k_1,k_2,k_3$ be positive integers. Let $P(k_1,k_2;k_3)$ be the digraph formed by three internally disjoint paths between two vertices $x,y$, two $(x,y)$-dipaths, one of size at least $k_1$, the other of size at least $k_2$, and one $(y,x)$-dipath of size at least $k_3$.
When we want to insist on the vertices $x$ and $y$, we denote it by $P_{xy}(k_1,k_2;k_3)$.


The {\it connected components} of a digraph are the connected components of its underlying graph.

Let $c$ be a colouring of a graph $G$. A subset of vertices or a subgraph $S$ of $G$ is {\it rainbow-coloured} if all vertices of $S$ have distinct colours.


A (directed) graph $G$ is {\it $k$-degenerate} if every subgraph $H$ of $G$ has a vertex of degree at most $k$.
The following proposition is well-known.
\begin{proposition}[useful??]\label{prop:deg}
Every $k$-degenerate (directed) graph is $(k+1)$-colourable.
\end{proposition}





We will need the following lemmas:

\begin{lemma}[useful??]\label{walk}
Let $D$ be a digraph and $W$ a walk of length $t^l$ such that every vertex appears at most $t$ times.
Then there exists a dipath of length $l$ starting with the first vertex of $W$.
\end{lemma}

\begin{proof}
We do it by induction on $l$. Let $x$ be the first vertex of $W$, because $x$ appears at most $t$ times, there exists a subpath of length greater
than $t^l/t$ not containing $x$. If we apply induction to the longest of such path, then we find a path of length $l-1$ without $x$ such that
 the first vertex is an out-neighbour of $x$.

\end{proof}

\begin{lemma}\label{min}
Let $\sigma=(u_t)_{t\in [p]}$ be a sequence of integers in $[k]$. If $p\geq l^k$, then there exists a set $L$ of $l$ indices such that for any $i,j \in L$ with $i < j$ the following holds : $u_i=u_j$ and $u_k > u_i$, for all $i < k < j$.
\end{lemma}

\begin{proof}
By induction on $k$, the result holding trivially when $k=1$. Assume now that $k>1$. Let $L_1$ be the elements of the sequence with value $1$. If $L_1$ has at least $l$ elements, we are done.
If not then there is a subsequence $\sigma'$ of $l^k/l$ consecutive elements in $\{2, \dots , k-1\}$. Applying the induction hypothesis to $\sigma'$ yields the result.
\end{proof}

\begin{lemma}\label{max}
Let $\sigma=(u_t)_{t\in [p]}$ be a sequence of integers in $[k]$. 
If $p > k (m-1)$, then there exists a subsequence of $m$ consecutive integers such that the
last one is the largest.
\end{lemma}

\begin{proof}
By induction on $k$, the result holding trivially when $k=1$. 
Let $i$ be the smallest integer such that $u_t\leq k-1$ for all $i\leq t\leq p$.
If $i>m$, then $u_{i-1}=k$, and the subsequence of the $i-1$ first elements of $\sigma$ is the desired sequence.
If $i\leq m$, apply the induction on $\sigma'=(u_t)_{i\leq t\leq p}$ which is a sequence of more than $(k-1)(m-1)$ integers in $[k-1]$, to get the result. 
\end{proof}

\begin{lemma}\label{lem:contrac}
Let $D$ be a digraph, $D_1 \dots D_l$ be disjoint subdigraphs of $D$ and $D'$ the digraph obtained by contracting all $D_i$ into
one vertex $d_i$. Then $\chi(D) \leq \chi(D')\cdot max_{i \in [l]}(\chi(D_i))$.
\end{lemma}

\begin{proof}

Let $k_1 = max_{i \in [l]}(\chi(D_i))$ and $k_2 = \chi(D')$ and. For each $i$, let $c_i$ be the colouring of $D_i$ using colours in $[k_1]$ and let
$c'$ be the colouring of $D'$ using colours in $[k_2]$. 
Define $c : V(D) \rightarrow ([k_1],[k_2])$ as follows. If $x$ is a vertex belonging to some $D_i$, then $c(x) = (c_i(x), c'(d_i))$, else $c(x) =(1,c'(x))$. 
Let $x$ and $y$ be adjacent vertices of $D$. If they belong to the same subdigraph $D_i$, then $c_i(x) \not = c_i(y)$ and $c(x) \not = c(y)$. If they don't belong
to the same component, then the vertices corresponding to these vertices in $D_{\mathcal{C}}$ will be adjacent and $c(x) \not = c(y)$. 
Thus $c$ is a proper colouring of $D$ using $(k+1)(2k)$ colours. 
\end{proof}



\section{Negative side}

Let $P_3$ be the union of three internally disjoint $(x,y)$-dipaths $Q_1$, $Q_2$ and $Q_3$ for some vertices $x$ and $y$ 
and $P_{2,2}$ be the union of four internally disjoint dipaths: two $(x,y)$-dipaths and two $(y,x)$-dipaths for some $x$ and $y$. 

\begin{theorem}
For every integer $k$, there exists a strongly connected digraph $D$ such that $\chi(D) >k$ and without subdivision of either $P_3$ or $P_{2,2}$. 
\end{theorem}

\begin{proof}
Let $D_{k,4}$ a digraph with chromatic number greater than $k$ and with every cycle having at least 4 blocks obtained by Theorem \ref{dkb}
Let $S = \{s_1, \dots,  s_l\}$ be the set of vertices of $D_{k,4}$ with outdegree 0 and $T = \{t_1, \dots, t_f\}$ the set of vertices with indegree 0.

Consider the digraph $D$ obtained from $D_{k,4}$ as follows. Add a dipath  $P = x_1,x_2....x_l,z,y_1,y_2,\dots, y_f$ and the arcs $s_ix_i$ and
$y_jt_j$. It is easy to see that $D$ is strongly connected.
Moreover, in $D$, every directed cycle uses the arc $x_lz$, therefore $D$ does not contain a subdivision of $P_{2,2}$, which has two arc-disjoint cycle. 
Suppose $D$ has a subdivision of $P_3$ between two vertices $u$ and $v$. Every vertex of $P$ has indegree and outdegree at most 2, meaning that $u$ and $v$ do not
belong to $P$. In $D$ each cycle on two blocks between vertices outside $P$ must use the arc $x_lz$. The union of $Q_1$ and $Q_2$ form a cycle on two blocks, 
which means one of the two paths, say $Q_1$ contains $x_lz$. But $Q_2$ and $Q_3$ also form a cycle on two blocks, but they cannot contain $x_lz$, a contradiction.
\end{proof}




\section{$P(k,1;1)$}



A collection ${\cal C}$ of directed cycles is {\it nice} if all cycles of ${\cal C}$ have length at least $2k$, and
any two distinct cycles $C_i,C_j\in\mathcal C$ intersect on at most one vertex. When this intersection exists, we call it $x_{i,j}$.
A {\it component} of $\mathcal{C}$ is a connected component in the adjacency graph of the $\mathcal{C}$, where vertices correspond to cycles in $\mathcal{C}$
and two vertices are adjacent if the corresponding cycles intersect. 
Call $D_{\mathcal{C}}$ the digraph obtained from $D$ by contracting each component of $\mathcal{C}$ into one vertex. 

We will prove that every $P(k,1;1)$ subdivision-free strong digraph $D$ has bounded chromatic number in the following way:
Take ${\cal C}$ a maximal nice collection of cycles. 
We will prove that every component $S$ of ${\cal C}$ induces a digraph $D[S]$ on $D$ of bounded chromatic number. 
Then we will prove that, since it doesn't contains large directed cycle and it is strongly connected, $D_{\mathcal{C}}$ has bounded chromatic number,
which, by \ref{lem:contrac}, allows us to conclude. 

We will need the following lemma:



\begin{lemma}\label{lem:headphone}
Let ${\cal C}$ be a nice collection of cycle in a $P(k,1;1)$ subdivision-free digraph $D$ and $C$, $C'$ two cycles of the same
component $S$ of $\mathcal{C}$. There is no path $P$ between $C$ and $C'$ which uses only edges outside of $S$.  
\end{lemma}

\begin{proof}
We suppose there exists an $(x,y)$-dipath $P$ with $x \in C$ and $y \in C'$ and show that we can find a subdivision of $P(k,1;1)$. 
By definition of $S$ there exists a $(s(Q),t(Q))$-dipath $Q$ from $C$ to $C'$ using edges in $S$. By choosing $C$ and $C'$ such that 
$Q$ is as small as possible, then $s(Q) \not = y$ and $t(Q) \not = x$ (note that $s(Q)$ and $t(Q)$ can be the same vertex). 

%By definition of $S$ there exists a sequence of cycles $C_1, C_2, \dots C_l$ such that $C_1 = C_i$, $C_l = C_j$ and $x_{s,s+1}$ exists for $s$ between $1$ and $l-1$.
On $C$, either $C[t(Q), x]$ or $C[x, t(Q)]$ is longer than $k$. 
%Let $P$ be the path from $x_{l-1,l}$ to $x_{1,2}$ using, outside of its ends, only vertices in $C_s$ for $s$ between $2$ and $l-1$ (if $l = 2$ then
%$x_{l-1,l} = x_{1,2}$ and $P = x_{1,2}$ works).
\begin{itemize}
	\item If $C[t(Q), x]$ has length at least $k$ then the union of $Q \odot C[t(Q), x] \odot P$, $C'[s(Q), y]$ and $C'[ y, s(Q)]$
	is a subdivision of $P(k,1;1)$ between $s(Q)$ and $y$. 
	\item If $C[x, t(Q)]$ is longer than $k$ then the union of $C[x, t(Q)]$, $P \odot C'[y, s(Q)] \odot Q$ and $C[t(Q), x]$
	is a subdivision of $P(k,1;1)$ between $x$ and $t(Q)$. 
\end{itemize}
\end{proof}

 
\begin{lemma}\label{lem:compo1}
Let ${\cal C}$ be a nice collection of cycles in a $P(k,1;1)$-subdivision-free digraph $D$ and $S$ a component of $\mathcal{C}$, then $\chi(D[S]) < 2k-1$.	 
\end{lemma}

\begin{proof}
By induction on the number of cycle in $S$. Let $C$ be a cycle of $S$. Since $C$ has length at least $2k$, there is no chords between $x$ and $y$ in $C$
such that $C[x,y]$ is longer than $k$. This means that $C$ is $2k-2$ degenerate and has chromatic number smaller than $2k -2$. Let $c$ be a colouring of $C$ with
$k$ colours. Let $S_1, S_2, \dots, S_r$ be the components containing cycles of $S$ in $\mathcal{C} \setminus{C}$. Since $S$ is exactly the union of the $S_i$ and $C$
then each $S_i$ has strictly less cycle than $S$. By induction there exists a colouring $c_i$ using $2k-1$ colours for each $S_i$.
 
Now we claim that the intersection of each $D[S_i]$ with $C$ is exactly one vertex. It is easy to see that $C$ must intersect at least one cycle of each $S_i$, 
now suppose there exists two vertices of $C$, $x$ and $y$ who belongs to $D[S_i]$. By definition of a nice collection, they cannot belong to the same cycle 
of $S_i$, so there exists two cycles $C_i$ and $C_j$ of $S_r$ such that $x \in C_i$ and $y \in C_j$. By lemma \ref{lem:headphone} where $C[x,y]$ plays the role
of $P$, it is not possible. This means, modulo a permutation of the $c_i$,
that we can consider that each vertex of $C$ receives the same colour in $c$ and in the $c_i$. 

Moreover, because of lemma \ref{lem:headphone}, there is no arc between different $S_i$ and between $S_i$ and $C$. This means that if we take the union of the 
$c_i$ and $c$, we obtain a proper colouring of $D[S]$ using $k+1$ colours. 
\end{proof}


\begin{lemma}\label{lem:reduce}
Let ${\cal C}$ be a maximal nice collection of cycle in a $P(k,1;1)$ subdivision-free strongly connected digraph $D$. Then $\chi(D_{\mathcal{C}}) < 2k$
\end{lemma}

\begin{proof}
First note that, since $D$ is strongly connected,  so is $D_{\mathcal{C}}$. Suppose $\chi(D_{\mathcal{C}}) \geq 2k$, by Bondy's Theorem, there exists a directed cycle
$C = x_1\dots x_l$ of length at least $2k$ in $ D_{\mathcal{C}}$. We deduce a cycle $C'$ in $D$ the following way:
Suppose the vertex $x_j$ corresponds to a component $S_i$ in $D$: the arc $x_{j-1}x_j$ corresponds in $D$ to an arc arriving on 
$p_i$, a vertex of $S_i$ and the arc $x_jx_{j+1}$ corresponds to an arc leaving $l_i$, a vertex of $S_i$. Let $P_j$ be a dipath 
from $p_i$ to $l_i$ in $D[S_i]$. Note that $P_j$ intersects each cycle of $S_i$ on a, possibly empty, subpath of $P_j$. 
Then $C'$ is the cycle obtained from $C$ by replacing the vertices $x_i$ which correspond to a component of $\mathcal{C}$ in $D$ by the path $P_j$.

$C'$ is a cycle of $D$ which is longer than $2k$ because it is longer than $C$. Let $C_1$ be a cycle of $\mathcal{C}$, then by construction of $C'$ 
and $D_{\mathcal{C}}$, $C'$ and $C_1$ can intersect only along a subpath of one $P_j$. Suppose this intersection is more than just a vertex and call $x$ and
$y$ the first and last vertices of this intersection. Then we find a subdivision of $P(k,1;1)$ between $x$ and $y$. The first is $C'[x,y]$ and the other are
$C[x,y]$ and $C[y,x]$.

So $C'$ is a cycle longer than $2k$, intersecting cycles of $\mathcal{C}$ on at most one vertex, and which does not belong to $\mathcal{C}$, or it would be reduced
to one vertex in $D_{\mathcal{C}}$, which contradicts the fact that $\mathcal{C}$ is maximal.
\end{proof}

So we can finally prove Theorem \ref{th:P11k}

\begin{proof}
Let ${\cal C}$ be a maximal nice collection of cycle in $D$, then lemmas \ref{lem:compo1}, \ref{lem:reduce} and \ref{lem:contrac} give the result.
\end{proof}


\section{$P(k,1;k)$}
%%%%%%%%%%%



We prove the result by the contrapositive. We consider a digraph $D$ that contains no subdivision of $P(k,1;k)$.
We shall prove that $\chi(D) \leq \cste$.

\subsection{Suitable collections of directed cycles}
%%%%%%%%%%%%%%%%%%%%%%%%%
 
 A collection ${\cal C}$ of directed cycles is {\it suitable} if all cycles of ${\cal C}$ have length at least $12k$, and for
any two distinct cycles $C_i,C_j\in\mathcal C$ intersect on  subpath $P_{i,j}$ of order at most $k$.
We denote by $s_{i,j}$ (resp. $t_{i,j}$) the initial (resp. terminal) vertex of  $P_{i,j}$. This notion generalises 
the notion of nice collection seen before. 
 
 
 
 \begin{lemma}\label{lem:dis}
 Let ${\cal C}$ be a suitable collection of directed cycles in a $P(k,1;k)$-\free\ digraph.
Let $C_1,C_2,C_3\in\mathcal C$ pairwise intersect, and let $v$ belong to $V(C_2)\cap V(C_3)\setminus V(C_1)$. Then exactly one of the following holds:
\begin{itemize}
\item $C_2[t_{1,2}, v]$ and $C_3[t_{1,3}, v]$ have both length at most $4k$; 
\item $C_2[v, s_{1,2}]$ and $C_3[v, s_{1,3}]$ have both length at most $4k$.
\end{itemize}
\end{lemma}

\begin{proof}
Suppose it is not the case. By symmetry, we may assume that $C_3[t_{1,3},v]$ has length at most $4k$ and
$C_2[v, s_{1,2}]$ has length at most $4k$, so  $C_2[t_{1,2},v]$ has length greater than $4k$.

Observe that $t_{2,3}\notin V(C_1)$ for otherwise $C_3[v,t_{1,3}]$ has length less than $2k$ (since it is contained in $P_{2,3}\cup P_{1,3}$ and each of those paths has length at most $k-1$ by definition of ${\cal C}$) and so $C_3[t_{1,3},v]$ has length at least $10k$, a contradiction.
We distinguish two cases according to the position of $s_{1,3}$ in $C_1$ relatively to $s_{1,2}$ and $t_{1,2}$.

\begin{itemize}
	\item Assume $s_{1,3}\in C_1[s_{1,2},t_{1,2}]$. The dipath  $C_3[s_1,3,t_{2,3}]$ is contained in $P_{1,2}\cup P_{2,3}$ and has lentgh less than $2k$. So $C_3[t_{2,3}, s_{1,3}]$ has length at least $k$. Moreover, $C_1[s_{1,3}, t_{1,2}]\odot C_2[t_{1,2}, t_{2,3}]$ contains $C_1[t_{1,2}, v]$  and so has legth at least $4k$. Consequently, the union of $C_1[s_{1,3}, t_{1,2}]\odot C_2[t_{1,2}, t_{2,3}]$, $C_3[t_{2,3}, s_{1,3}]$, and
$C_2[t_{2,3}, s_{1,2}]\odot C_1[s_{1,2}, s_{1,3}]$ is a subdivision of $P(k,1;k)$, a contradiction.

	\item Assume now that $s_{1,3}\in C_1[t_{1,2},s_{1,2}]$. The dipath  $C_2[s_{1,2}, t_{2,3}]$ has length greater than $4k$ because it contains $C_2[t_{1,2},v]$. Moreover, the dipath $C_3[s_{1,3}, t_{2,3}]$ is contained in $P_{1,3}\cup C_3[t_{1,3},v]$, and so has length at most $5k$.
	Thus $C_3[t_2,3, s_{1,3}]$ has length at least $7k$.
	 Consequently, the union of $C_2[s_{1,2}, t_{2,3}]$, $C_3[t_2,3, s_{1,3}]\odot C_1[s_{1,3}, s_{1,2}]$, and $C_2[t_{2,3}, s_{1,2}]$ is a subdivision of $P(k,1;k)$, a contradiction.
\end{itemize}
\end{proof}
 
 
 
 
 
 
% Let $D[\mathcal{C}]$ be the graph induced on $D$ by the vertices in $\mathcal{C}$ and let $D'$ be the subdigraph of $D[\mathcal{C}]$ with vertex set $V(D)$ and arc set  $\bigcup_{C\in {\cal C}} A(C)$.

Let ${\cal C}$ be a suitable collection of directed cycles.
For every set of vertices or digraph $S$, we denote by ${\cal C}\cap S$ the set of cycles of ${\cal C}$ that intersect $S$. 



Let $C_1\in \mathcal{C}$.
For each $C_j\in {\cal C} \cap C_1$, let $Q_j$ be the subdipath of $C_j$ 
containing all the vertices that are at distance at most $4k$ from $P_{1,j}$ in the cycle underlying $C_j$.
Then the dipath $C_j[s(Q_j), s_{1,j}]$ and  $C_j[t_{1,j},t (Q_j)]$ have length $4k$.
Set $Q^-_j=C[s(Q_j), s_{1,j}[$ and $Q^+_j=C]t_{1,j}, t(Q_j)]$.


Set $I^+(C_1) = \bigcup_{C_j\in {\cal C}\cap C_1} Q^+_j$ and $I^-(C_1) = \bigcup_{C_j\in {\cal C}\cap C_1} Q^-_j$.
Observe that Lemma~\ref{lem:dis} implies that $I^+(C_1)$ and $I^-(C_1)$ are vertex-disjoint digraphs. We set $I(C_1)=I^+(C_1)\cup I^-(C_1)$.




\begin{lemma}\label{lem:A}
Let ${\cal C}$ be a suitable collection of directed cycles in a $P(k,1;k)$-\free\ digraph.
Let $C_1$ be a cycle of ${\cal C}$ and let $A$ be a connected component of $\bigcup{\cal C} - I(C_1)$.
All vertices of $({\cal C}\cap A)  - A$ belongs to a unique cycle $C_A$ of $\mathcal{C}$.
\end{lemma}

\begin{proof}
Suppose it is not the case. Then there are two distinct cycles $C_2, C_3$ of ${\cal C}\cap A$ that intersect with $C_1$.

Let $t$ be the closest vertex among $\{t_{1,2}, t_{1,3}\}$ in $C_1$ and let $t'$ be the other vertex of $\{t_{1,2}, t_{1,3}\}$.
(If $t_{1,2}=t_{1,3}$, then $t=t'$.) For $i=2,3$, let $R_i$ be the $(t,t(Q_i)$-dipath in $C_1\cup Q_i$. Observe that $R_i=Q_i$ or $R_i=C_1[t,t']\cup Q_I$. Hence $R_i$ has length at least $4k$.
Let $v$ be the last vertex in $R_2\cap R_3$ along $R_2$. (This vertex exists since $t\in R_2\cap R_3$.) In $A$, there exists a $(t(Q_3), C_2)$-dipath $R_A$. Let $w$ be its terminal vertex. By definition of $A$, $w$ is in $C_2[t(Q_2, s(Q_2)]$, therefore $C_2[w,v]$ has length at least $4k$ since it contains $C_2[s(Q_2), s_{1,2}]$. Consequently, both $R_2[v,t(Q_2)]$ and $R_3[v,t(Q_3)]$ have length less than $k$ for otherwise the union of $C_2[w,v]$, $C_2[v,w]$ and $R_3[v,t(Q_3)]\odot R_A$ would be a subdivision of $P(k,1;k)$.
In particular, $v\neq t$. This implies that $s_{2,3}\in V(R_2\cap R_3)$.
Moreover, $R_2[s_{2,3}, t(Q_2)]$ has length at most $2k$ because $R_2[s_{2,3},v]$ is a subdipath of $P_{2,3}$ and so has length at most $k$.
Therefore $R_2[t,s_{2,3}]$ has length at least $2k$ because $R_2$ has length at least $4k$.
It follows that the union of $C_2[s_{2,3},t]$,  $R_2[t,s_{2,3}]$ and  $R_3[t,s_{2,3}]$ is a subdivision of $P(k,1;k)$, a contradiction.
\end{proof}




\begin{lemma}\label{lem:no-dicycle}
Let ${\cal C}$ be a suitable collection of directed cycles in a $P(k,1;k)$-\free\ digraph. For any cycle $C_1\in {\cal C}$, the digraph
$I^+(C_1)$ has no directed cycle.  
\end{lemma}
\begin{subproof}
Suppose for a contradiction that $I^+(C_1)$  contains a directed cycle $C'$.
Clearly, it must contain arcs from at least two $Q^+_j$.



Assume that $C'$ contains several vertices of $Q^+_j$.
Necessarily, there must be two vertices $x,y$ of $Q^+_j\cap C'$ such that no vertex of $C']x,y[$ is in $C_j$ and  $y$ is before $x$ in $Q^+_j$.
Therefore $C'[x,y]\odot Q^+[y,x]$ is also a directed cycle in $I^+(C_1)$. Free to consider this cycle,
we may assume that $C'\cap Q^+_j$ is a directed path.

Doing so, for all $j$, we may assume that $C'\cap Q^+_j$ is a directed path  for every $C_j\in {\cal C}\cap C_1$.
Without loss of generality, we may assume that there are cycles $C_2, \dots , C_p$ such that
\begin{itemize}
\item $C'$ is in $Q^+_2\cup \cdots  \cup Q^+_p$;
\item for all $2\leq j\leq p$, $C'\cap Q^+_j$ is a directed path $P^+_j$ with initial vertex $a_j$ and terminal vertex $b_j$;
\item the $a_j$ and the $b_j$ appear according to the following order around $C'$: $(a_2, b_p, a_3, b_2, \dots ,   a_p, b_{p-1}, a_2)$ with possibly $a_{j+1}=b_j$ for some $1\leq j \leq p$ where $a_{p+1}=a_2$.
\end{itemize}
For $2\leq j\leq p$, set $B_j=C_j[b_j, a_j]$. Note that $B_j$ has length at least $7k$, because $Q^+_2$ has length at most $5k$.


Consider the closed directed walk $$W=C_p[a_2,b_p]\odot B_p \odot C_{p-1}[a_p, b_{p-1}] \odot  \cdots  \odot  B_3\odot C_2[a_3, b_2]\odot B_2.$$
$W$ contains a directed cycle $C_W$. Wihtout loss of generality, we may assume that this cycle is of the form
$$C_W=B_q[v, a_q] \odot C_{q-1}[a_q, b_{q-1}] \odot  \cdots  \odot  B_3\odot C_2[a_3, b_2]\odot B_2[b_2, v]$$
for some vertex $v\in B_2\cap B_r$. (The case when $W$ is a directed cycle corresponds to $q=p+1$ and $B_2=B_{p+1}$.)

Note that necessarily, $q\geq 4$, for $B_3$ does not intersect $B_2$, for otherwise $b_3=b_2$ since the intersection of $C_2$ and $C_3$ is a dipath.

Observe that $C_W[b_2,v]=C_2[b_2,v]$ or $C_W[v, a_4]$ has length at least $k$.
Indeed, if $q=p+1$, then it follows from the fact that $B_2$ has length as least $7k$; if
$5\leq q\leq p$, then it comes form the fact that $B_4$ is a subdipath of $C_W[v, a_r]$; if $q=4$, then it follows from Lemma~\ref{lem:dis}  applied to $C_3$, $C_2$, $C_4$ in the role of $C_1$, $C_2$, $C_3$ respectively.
In both case, $C_W[b_2, a_4]$ has length at least $k$. 

Furthermore, $C_W[a_4,b_2]$ has length at least $k$ because it contains $B_3$. Therefore the union of
$C_W[b_2, a_4]$, $C_W[a_4,b_2]$ and $C'[b_2,a_4]=C_3[b_3,a_4]$ is a subdivision of of $P(k,1;k)$, a contradiction.
\end{subproof}




\begin{lemma}\label{lem:IC}
Let ${\cal C}$ be a suitable collection of directed cycles in a $P(k,1;k)$-\free\ digraph.

Let $c$ be a partial colouring of a cycle $C_1\in {\cal C}$ such that only a path of length at most
$9k$ is coloured and this path is rainbow-coloured. Then $c$ can be extended into
a colouring of $C_1\cup I(C_1)$ using $2\cdot \col + 18k$ colours, such that every subpath of length at most $9k$ of $C_1$ is rainbow-coloured and  $Q_j$ is rainbow-coloured, for every $C_j\in {\cal C}\cap C_1$.\end{lemma}

\begin{proof}
We can easily extend $c$ to $C_1$ using $18k$ colours (including the at most $9k$ already used colours)
so that every subpath of $C_1$ of length $9k$ is rainbow-coloured.

We shall now prove that there exists a colouring $c^+$ of $I^+(C_1)$ with $\col$ (new) colours so that $Q^+_j$ is rainbow-coloured for every $C_j\in {\cal C}\cap C_1$, and a colouring $c^-$ of $I^-(C_1)$ with $\col$ (other new) colours so that $Q^-_j$ is rainbow-coloured for every $C_j\in {\cal C}\cap C_1$.
The union of these three colourings is clearly the desired colouring of $C_1\cup \bigcup{\cal C}[I(C_1)]$. (Observe that a vertex of $I(C_1)$ is coloured only once because  $I^+(C_1)$ and $I^-(C_1)$ are disjoint.)

It remains to prove the existence of $c^+$ and $c^-$. By symmetry, it suffices to prove the existence of $c^+$.
 To prove the existence of $c^+$ we consider an auxilary digraph $D^+_1$.
 For each $C_j\in {\cal C}\cap C_1$, let $T^+_j$ be the transitive tournament whose hamiltonian dipath is $Q^+_j$.
Let $D^+_1=\bigcup_{C_j\in  {\cal C}\cap C_1} T^+_j$. 
The arcs of the $A(T^+_j)\setminus A(Q^+_j$ are called {\it fake arcs}.
%Observe that $D^+_1$ is acyclic because $I^+(C_1)$ is. Indeed, if there were a directed cycle
%cycle in $D^+_1$, it could be transformed into a closed directed walk (by replacing each arc $(u,v)\in A(T^+_j)$, by 
%$Q^+_j[u,v]$), which would contain a directed cycle.
Clearly, $c^+$ exists if and only if $D^+_1$ admits a proper colouring. Henceforth it remains to prove the following claim.

\begin{claim}
$\chi(D^+_1) \leq  \col$.
\end{claim}

\begin{subproof}
To each vertex $v$ in $I^+(C_1)$ we associate the set $\Dis(v)$ of the lengths of the $C_j[t_{1,j},v]$ for all cycle $C_j\in {\cal C} \cap C_1$ containing $v$ such that $C_j[t_{1,j},v]$ has length at most $4k$.


Suppose for a contradiction that $\chi(D^+_1) \leq  \col$.
By Gallai-Roy Theorem, $D^+_1$ admits a directed path of length $\col$. Replacing all fake arcs  $(u,v)$ in some $A(T^+_j)$, by 
$Q^+_j[u,v]$ we obtain a directed walk $P$ in $I^+(C_1)$ of length at least $\col$. By Lemma \ref{lem:no-dicycle}, $P$ is necessarily a directed path. Set $P=(v_1,  \dots , v_{p})$. We have $p\geq \col$.

For $1\leq i\leq p$, let $m_i = \min \Dis(v_i)$. 
Lemma \ref{min} applied to $(m_i)_{1\leq i\leq p}$ yields  a set of $8k^2$ indices of such that 
for any $i< j \in L$,  $m_i=m_j$ and $m_k > m_i$, for all $i< k < j$.
Let $l_1 < l_2 < \cdots < l_{8k^2}$ be the elements of $L$ and let $m= m_{l_1} = \cdots = m_{l_{8k^2}}$.


For $1\leq j\leq 8k^2-1$, let $M_j = \max \bigcup_{l_j\leq i < l_{j+1}} \Dis(v_i)$.
By definition $M_j\leq 4k$.
Applying Lemma~\ref{max} to $(M_j)_{1\leq j\leq 8k^2}$,  we get a sequence of size $2k$ $M_{j_0+1} \dots M_{j_0+{2k}}$ such that $M_{j_0+{2k}}$ is the greatest. For sake of simplicity, we set $\ell_i =j_0+i$ for $1\leq i\leq 2k$.
Let $f$ the smallest index not smaller than $\ell_{2k}$ for which $M_{\ell_{2k}} \in \Dis (v_f)$. 

Let $j_1$ be an index such that $C_{j_1}[t_{1,j_1},v_{\ell_1}]$ has length $m$ and set $P_1=C_{j_1}[t_{1,j_1},v_{\ell_1}]$.
Let $j_2$ be an index such that $C_{j_2}[t_{1,j_2},v_{\ell_k}]$ has length $m$ and set $P_2=C_{j_2}[t_{1,j_2},v_{\ell_k}]$.
Let $j_3$ be an index such that  $C_{j_3}[t_{1,j_3},v_{f}]$ has length $M_{\ell_{2k}}$ and set $P_3=C_{j_3}[v_f, s_{1,j_3}]$ (some vertices of $P_3$ are not in $I^+(C_1)$).

Note that any internal vertex $x$ of $P_1$ or $P_2$ has an integer in $\Dis(x)$
which is smaller than $m$ and every internal vertex $y$ of $P_3$ has an integer in $\Dis(y)$ which
is greater than $M_{\ell_{2k}}$, or does not belong to $I^+(C_1)$. Hence, 
$P_1$, $P_2$ and $P_3$ are disjoint from $P[v_{\ell_1},v_f]$. 


We distinguishes between the intersection of $P_1$, $P_2$ and $P_3$:

\begin{itemize}
	\item Suppose $P_3$ does not intersect $P_1 \cup P_2$.
	\begin{itemize}
		\item Assume first that $P_1$ and $P_2$ are disjoint. If $s(P_1)$ is in $C_1[t(P_3), s(P_2)]$, then the union of  $P_1 \odot P[v_{\ell_1}, v_{\ell_k}]$, $P[v_{\ell_k}, v_f] \odot P_3\odot C_1[t(P_3), s(P_1)]$ and $C_1[s(P_1), s(P_2)]\odot P_2$ is a subdivision of $P(k,1;k)$, a contradiction.
		 If $s(P_1)$ is in $C_1[ s(P_2), t(P_3)]$, then the union of  $C_1[s(P_2), s(P_1)]\odot P_1\odot P[v_{\ell_1}, v_{\ell_k}]$, $P[v_{\ell_k}, v_f]\odot  P_3\odot C_1[t(P_3), s(P_2)]$, and  $P_2$ is a subdivision of $P(k,1;k)$, a contradiction.
	
		\item Assume now $P_1$ and $P_2$ intersect. Let $u$ be the last vertex along $P_2$ on which they intersect. The union of $P_1[u,v_{\ell_1}]\odot P[v_{\ell_1}, v_{\ell_k}]$, $P[v_{\ell_k}, v_f]\odot P_3\odot C[t(P_3), s(P_1)]\odot P_1[s(P_1), u]$, and $P_2[u, v_{\ell_k}]$ is a subdivision of $P(k,1;k)$, a contradiction.
	\end{itemize}

	\item Assume $P_3$ intersect $P_1\cap P_2$. Let $v$ be the first vertex along $P_3$ in $P_1\cap P_2$ and let $u$ be the last vertex of $P_1\cap P_2$ along $P_2$. The union of $P_1[u,v_{\ell_1}]\odot P[v_{\ell_1}, v_{\ell_k}]$, $P[v_{\ell_k}, v_f]\odot P_3[v_f,v]\odot P_1[v, u]$, and $P_2[u, v_{\ell_k}]$ is a subdivision of $P(k,1;k)$, a contradiction.

	\item Assume now that $P_3$ intersect $P_1\cup P_2$ but not $P_1\cap P_2$. Let $v$ be the first vertex along $P_3$ in $P_1\cup P_2$.
	\begin{itemize}
	\item If $v \in P_2$, let $u$ be the last vertex on $P_2\cap P_3$ along $P_3$. Observe that $P_3[v,u]$ is also a subpath of $P_2$ and therefore contains no vertex of $P_1$. Furthermore, there is a dipath $Q$ from $u$ to $v_{\ell_1}$ in $P_3[u, t(P_3)]\cup C_1\cup  P_1$. Hence, the union of $P[v_{\ell_k}, v_f] \odot P_3[v_f,v]$, $Q\odot P[v_{\ell_1},v_{\ell_k}]$, and $P_2[u,v_{\ell_k}$ is a subdivision of $P(k,1;k)$, a contradiction.
	
	\item If $v\in P_1$, let $u$ be the last vertex on $P_1\cap P_3$ along $P_3$.  Observe that $P_3[v,u]$ is also a subpath of $P_1$ and therefore contains no vertex of $P_2$. Furthermore, there is a dipath $Q$ from $u$ to $v_{\ell_k}$ in $P_3[u, t(P_3)]\cup C_1\cup  P_2$.
	The union of $P[v_{\ell_k}, v_f] \odot P_3[v_f,u]$, $P_1[u, v_{\ell_1}]\odot P[v_{\ell_1}, v_{\ell_k}]$ and $Q$ is a subdivision of $P(k,1;k)$, a contradiction.

	\end{itemize}

\end{itemize}
\end{subproof}
\end{proof}




\begin{lemma}\label{lem:col-union-cycle}
Let ${\cal C}$ be a suitable collection of directed cycles in a $P(k,1;k)$-\free\ digraph.
There exists a proper colouring $c$ of $\bigcup{\cal C}$ with $2\cdot \col + 18k$ colours, such that, each subpath of length $9k$ of each cycle of $\mathcal{C}$ is rainbow-coloured.
\end{lemma}
\begin{proof}
We prove by induction on the number of cycles in $\mathcal{C}$ the following stronger statement:
{\it if there exists
a partial colouring $c$ such that one of the cycle $C_1$ has a path of length less than $9k$
which is rainbow-coloured, then we can extend this colouring to all $D[\mathcal{C}]$ using less
than $2\col +18k$ colours such that, on each cycle, every subpath of length
$9k$ is rainbow-coloured}.

Consider a rainbow-colouring of a subpath of length less than $9k$ of a cycle $C_1\in {\cal C}$.
By Lemma \ref{lem:IC}, we can extend this colouring to a colouring $c_1$ of $I(C_1)$ at most $2\cdot \col + 18k$ colours.
Note that the non-coloured vertices of $\bigcup{\cal C}$ are in one of the connected components of $\bigcup{\cal C} - I(C_1)$.
Let $A$ be a connected component of $\bigcup{\cal C} - I(C_1)$. The coloured (by $c_1$) vertices of ${\cal C}\cap A$ are those of $({\cal C}\cap A)-A$. Hence, by Lemma \ref{lem:A}, they all belong to some cycle $C_j$ and so to the diptah $Q_j$ which has length at most $9k$.
Hence, by the induction hypothesis, we can extend $c_1$ to $A$. Doing this for each component, we extend $c_1$ to the whole $\bigcup{\cal C}$.
\end{proof}





\begin{lemma}\label{lem:DC}
Let ${\cal C}$ be a suitable collection of directed cycles in a $P(k,1;k)$-\free\ digraph $D$.
For every strongly connected component $\mathcal{S}$ of $\bigcup \mathcal{C}$, 
$\chi(\mathcal{S}) < (2\cdot \col + 18k) \cdot (2 \cdot \dr)$.
\end{lemma}

Consider the colouring $c$ obtained by lemma \ref{lem:col-union-cycle}. It allows us 
to deal with the arcs belonging to cycles in $\mathcal{C}$, to deal with the other, we do the following:
We present a sort BFS for $\mathcal{S}$.
Let $C_r$ (for root) be a cycle of $\mathcal{S}$ and $L_0 = \{C_r\}$. For every cycle $C_s$ of $\mathcal{S}$ for which 
$P_{r,s}$ is not empty, we put $C_s$ in $L_1$ and say that $C_0$ is the \textit{father} of $C_s$. We build $L_i$ inductively
until we cover all cycle of $\mathcal{S}$ : $L_{i+1}$ consists of every cycle $C_l$ not in $\cup_{j \leq i} L_j$ such that 
there exists a cycle $C_a \in L_i$ such that $P_{a,l}$ is non empty. $C_a$ is the father of $C_l$ and it is important to note that we 
fix it uniquely, even though there might exists multiple choices for a father of $C_a$ in $L_i$. We define the \textit{ancestors} of 
cycle like in trees.


Let $C_l$ be a cycle and $C_a$ its father. 
%let $Q_l$ be the subdipath of $C_l$ which contains
%all the vertices that are at distance at most $k$ from $P_{l,a}$ in the cycle underlying $C_l$.
%Then the dipath $C_l[s(Q_l), s_{l,a}]$ and  $C_j[t_{a,l},t (Q_l)]$ have length $k$.
%Set $Q^-_l=C[s(Q_l), s_{l,a}[$ and $Q^+_l=C]t_{l,a}, t(Q_l)]$.

Let $Q^-_l$ be the set of vertices $x$ of $C_l \setminus C_a$ such that the path $C_l[t_{a,l}, x]$ is longer than $k$. 
We denote by $Q^+_l$ $C_l \setminus( C_a \cup Q^-_l) $. Note that for a vertex $x$ in  $Q^+_l$, $C_l[x,s_{a,l}]$ is longer than $k$. 

For a vertex $x$ belonging to cycles in $\mathcal{S}$, we say that $x$ \textit{belongs to}  $L_i$ if
$i$ is the smallest integer such that $x \ C_l$ for some $C_l \in L_i$.

\begin{lemma}\label{lem:Qpos}
Suppose there exists a vertex $x$ such that $x \in Q^+_l$ and $x \in C_s$ for some cycles $C_l$ and $C_s$ in some $L_j$.
If $x$ belongs to $L_j$ then  $x \in  Q^+_s$ or $D$ contains a subdivision of $P(k,1;k)$
\end{lemma}

\begin{proof}
Suppose $x$ belongs to $L_j$ and $x \not \in  Q^+_s$. We can assume $x = l_{s,l}$.
Let $C_{l'}$ and $C_{s'}$ be the father of $C_l$ and $C_s$ (they can be the same). 
By definition of the $L_i$ there exists a dipath $P$ from $s_{l,l'}$ to $l_{s,s'}$ only going through $C_{l'}$, $C_{s'}$ and their ancestors.
In particular $P$ is disjoint from $C_l \setminus C_{l'}$ and  $C_s \setminus C_{s'}$
Let $y$ be the first vertex of $P$ which belongs to $C_s$. Then we find a subdivision of $P(k,1;k)$ between $x$ and $y$.
The first path is $C_s[y,x]$ longer than $k$, the second is $C_l[x,s_{l,l'}] \odot P[ s_{l,l'}, y]$ and the last is $C_s[x,y]$.
\end{proof}

The same works for $Q^-$:

\begin{lemma}\label{lem:Qneg}
Suppose there exists a vertex $x$ such that $x \in Q^-_l$ and $x \in C_s$ for some cycles $C_l$ and $C_s$ in some $L_j$.
If $x$ belongs to $L_j$ then  $x \in  Q^-_s$ or $D$ contains a subdivision of $P(k,1;k)$
\end{lemma}


%\begin{lemma}\label{lem:Ql}
%Suppose there exists an arc between $x_1 \in C_{l_1}$ and $x_2 \in C_{l_2}$. If $C_{l_1}$ and $C_{l_1}$ both belong to the same
%$L_j$ then $x_1 \in Q_{l_1}$ and $x_2 \in Q_{l_2}$. 
%\end{lemma}

%\begin{proof}
%Suppose $x_1 \not \in  Q_{l_1}$. 
%Let $C_{s_1}$ and $C_{s_1}$ be the fathers of $C_{l_1}$ and  $C_{l_2}$.
%Let $P$ be the path from $P{l_1,s_1} to P{l_2,s_2}$ in the union of $L_t$ for $t<j$ and $x$
%and $y$ are the first and last vertices of this dipath. 
%We distinguishes between three different cases, regarding the intersection of $C_{l_1}$ and  $C_{l_2}$.
%\begin{itemize}
%	\item Suppose they don't intersect. 
%	Then we find a subdivision of $P(k,1;k)$ between $y$ and $x_1$. 
%	The long paths are $C_{l_1}[y,x_1]$ and $C_{l_1}[x_1,y]$ and the last is $(x_1,x_2) \odot C_{l_2}[x_2,y] \odot P$.
%	\item Suppose they intersect By lemmas \ref{lem:Qneg} and \ref{lem:Qpos}
%\end{itemize}
%
%\end{proof}
We define $Q^+(r)$ (resp $Q^-(r)$) a partition of the vertices coloured $r$ in $c$ as follow:
If $x$ belongs to $L_j$ and, for all cycle $C_l$ in $L_j$ such that $x \in C_l$, $x \in Q^+_l$ then $x \in Q^+(r)$.
If all cycle $C_l$ in $L_j$ such that $x \in C_l$, $x \in Q^-_l$ then $x \in Q^-(r)$.
By lemmas \ref{lem:Qpos} and \ref{lem:Qneg}, this is indeed a partition of the vertices coloured $r$ in $c$.
Thus, if for every $r$ we can colour the graph $D_r^+$ induced on $D$ by the vertices in 
$Q^+(r)$ with $\dr$ colours, then we can colour the graph induced by a $\mathcal{S}$ using 
$(2\cdot \col + 18k) \cdot (2 \cdot \dr)$ colours.  


\begin{lemma}\label{lem:lvl}
Let $x$ and $y$ be vertices of $Q^+(s)$ belonging to $L_i$ and $L_j$, such that $i-j> k$ then there is no arc between $x$ and $y$.
\end{lemma}

\begin{proof}
Let $C_x$ be the cycle of $L_i$ such that $x \in C_x$ and $C_y$ the cycle of $L_j$ such that $y \in C_y$.
By considering the common ancestor of $C_x$ and $C_y$, we obtain a sequence of cycle $C_1 \dots C_p$ such that $C_1 = C_x$ and $C_p = C_y$
and $P_{i,i+1}$ exists for all $i$ between 1 and $l-1$.
If the arc is from $x$ to $y$, then we find a subdivision of $P(k,1;k)$ between $l_{p-1,p}$ and $y$. 
First note that by definition of the $L_i$, the dipaths $C_{i-1}[l_{i-1,i}, l_{i-2,i-1}]$ are all disjoint. 
Then the path $P = C_{p-1}[l_{p-1,p}, l_{p-2,p-1}] \odot C_{p-2}[l_{p-2,p-1}, l_{p-3,p-2}] \dots \odot C_{2}[l_{2,3}, s_{2,1}] 
\odot C_1[s_{2,1}, x]$ is longer than $k$. The subdivision consists of $P \odot (xy)$, $C_l[l_{p-1,p}, y]$ and $C_l[y,l_{p-1,p}]$.
The case where the arc is from $y$ to $x$ is similar.
\end{proof}

\begin{lemma}
$\chi(D_r^+) < \dr$
\end{lemma}

\begin{proof}
We use the following partition of the arcs of $D_r^+$

\begin{itemize}
	\item $A_0$ is the set of arcs of $D_r^+$ which ends belong to different levels.
	\item $A_1$ is the set of arcs of $D_r^+$ which ends belong to the same levels.
\end{itemize} 

By lemma \ref{lem:lvl}, $\chi(D_r^+[A_0]) \geq k$. Indeed, assign to each level $L_i$ the colour $i$ modulo $k$ and
we obtain a proper colouring.

We will now prove that $\chi(D_r^+[A_1]) \geq k$ by contradiction. 
Suppose it is not, it means that there exists a level $i$ such that the chromatic number of $\chi(D_r^+[A_1])$ in this level is greater 
than $k$. This means there exists a dipath $P$ of length $k$ in this level between vertices $x$ and $y$. Suppose this path visits twice 
the same cycle $C_1$ of $\mathcal{C}$. Because all the points belonging to the path received colour $r$ in $c$ this means there exists a 
handle between two points at distance at least $9k$ of the cycle $C_1$ in $\mathcal{C}$, which is impossible. 
Consider $C_x$ and $C_y$ cycles containing $x$ and $y$ in $L_i$ and $C_{x'}$, $C_{y'}$ their respective father. 
By considering the common ancestor of $C_{x'}$ and $C_{y'}$, we can find a dipath $P_y$ from $l_{y,y'}$ to $s_{x,x'}$ which 
only uses vertices in levels smaller than $i-1$, thus disjoint of $P$, $C_y[l_{y,y'}, y]$, $C_y[y, l_{y,y'}]$ and $C_x[s_{x,x'}, x]$. 
Then $C_y[l_{y,y'}, y]$, $C_y[y, l_{y,y'}]$ and $P_y \odot C_x[s_{x,x'}, x] \odot P$ form a subdivision of $P(k,1;k)$ between $y$ 
and $l_{y,y'}$. 

By lemma \ref{lem:decomp} this means $\chi(D_r^+) <  \dr$ 

\end{proof}

\subsection{Proof of Theorem \ref{th:main}}

Consider $\mathcal{C}$ be a maximal suitable collection of cycles in $D$. Let $D'$ be the digraph obtained by contracting 
every connected component $S$ of $\bigcup \mathcal{C}$ into one vertex. For each connected component $S_i$ we call $s_i$ the
new vertex created.

\begin{claim}\label{cl:nocycle}
$\chi(D') \leq 12k$
\end{claim} 

\begin{proof}
First note that since $D$ is strongly connected so is $D'$. If $\chi(D') > 12k$ then by Bondy's Theorem, there exists
a directed cycle $C = x_1x_2 \dots x_l$ of length greater than $12k$. From $C$ we deduce a cycle $C'$ in $D$ in the following way: 
\begin{itemize}
	\item We keep the vertices of $D'$ which correspond to vertices in $D$.
	\item For each vertex $x_j$ which correspond to a $s_i$ in $D$: the arc $x_{j-1}x_j$ corresponds in $D$ to an arc arriving on 
		$p_i$, a vertex of $S_i$ and the arc $x_jx_{j+1}$ corresponds to an arc leaving $l_i$, a vertex of $S_i$. Let $P_j$ be the dipath 
		from $p_i$ to $l_i$ in $\bigcup \mathcal{C}$. Note that this path intersect the element of $S_i$ only along a subpath. 
\end{itemize}
$C'$ is then the path obtained from $C$ where we replace all contracted point $x_j$ by the path $P_j$. 
First note that $C'$ has size greater than $12k$. Moreover, let $C_i$ be a cycle of $\mathcal{C}$, then $C_i$ can
intersect $C'$ only along one $P_j$, because they all correspond to different component of $\bigcup \mathcal{C}$. This mean the
intersection of $C'$ to each cycle of $\mathcal{C}$ is along a subpath. If this intersection is along a subpath longer than $k$ than we can 
find a subdivision of $P(k,1;k)$. So $C'$ is a directed cycle of length at least $12k$ which intersect every cycle of $\mathcal{C}$ along a 
subdipath of length at most $k$. This contradicts the maximality of $\mathcal{C}$ and $\chi(D') \leq 12k$.
\end{proof}


Lemma \ref{lem:contrac} implies that we can colour $D$ using $12k \cdot (2\cdot \col + 18k) \cdot (2 \cdot \dr)$ colours.



\begin{thebibliography}{XX}

\bibitem{AHT07}
L.~Addario-Berry, F.~Havet, and S.~Thomass{\'e}.
\newblock Paths with two blocks in $n$-chromatic digraphs.
\newblock \emph{Journal of Combinatorial Theory, Series B}, 97
  (4): 620--626, 2007.

\bibitem{AHS+13}
L.~Addario-Berry, F.~Havet, C.~L. Sales, B.~A. Reed, and S.~Thomass{\'e}.
\newblock Oriented trees in digraphs.
\newblock \emph{Discrete Mathematics}, 313 (8): 967--974,
  2013.


\bibitem{alon2014coloring}
N. Alon, A. Kostochka, B. Reiniger, D.~B West, and X.
  Zhu.
\newblock Coloring, sparseness, and girth.
\newblock {\em arXiv preprint arXiv:1412.8002}, 2014.


 %\bibitem{BeWo83} A. Benhocine and A. P. Wojda.
 %\newblock On the existence of specified cycles in a tournament.
 %\newblock \emph{J. Graph Theory}, 7: 469-473, 1983.

   \bibitem{Bon76} J. A. Bondy, Disconnected orientations and a conjecture of Las Vergnas,
{\it J. London Math. Soc. (2)}, {\bf 14} (2) (1976), 277--282.


  \bibitem{BoMu08}
J.A. Bondy and U.S.R. Murty.
\newblock {\em {G}raph {T}heory}, volume 244 of {\em Graduate Texts in
  Mathematics}.
\newblock Springer, 2008.


%\bibitem{Bro41}
%R.~L. Brooks.
%\newblock On colouring the nodes of a network.
%\newblock {\em Proc. Cambridge Philos. Soc.}, 37:194--197, 1941.

\bibitem{Burr80}
S.~A. Burr.
\newblock Subtrees of directed graphs and hypergraphs.
\newblock In \emph{Proceedings of the 11th Southeastern Conference on
  Combinatorics, Graph theory and Computing}, pages 227--239, Boca Raton - FL,
  1980. Florida Atlantic University.

\bibitem{Bur82} S. A. Burr,
Antidirected subtrees of directed graphs.
{\it Canad. Math. Bull.} {\bf 25} (1982), no. 1, 119--120.

\bibitem{CHLN16}
N. Cohen, F. Havet, W. Lochet, and N. Nisse.
\newblock Subdivisions of oriented cycles in digraphs with large chromatic number.
\newblock arXiv:1605.07762

\bibitem{Erd59}
P.~Erd{\H{o}}s.
\newblock Graph theory and probability.
\newblock {\em Canad. J. Math.}, 11:34--38, 1959.

\bibitem{ErHa66}
P.~Erd{\H{o}}s and A. Hajnal.
\newblock On chromatic number of graphs and set-systems.
\newblock {\em Acta Mathematica Academiae Scientiarum Hungarica}, 17(1-2):61--99, 1966.


\bibitem{EL75}
P.~Erd{\H{o}}s and L.~Lov{\'a}sz.
\newblock Problems and results on {$3$}-chromatic hypergraphs and some related
  questions.
\newblock In {\em Infinite and finite sets ({C}olloq., {K}eszthely, 1973;
  dedicated to {P}. {E}rd{\H o}s on his 60th birthday), {V}ol. {II}}, pages
  609--627. Colloq. Math. Soc. J\'anos Bolyai, Vol. 10. North-Holland,
  Amsterdam, 1975.


\bibitem{Gal68}
T.~Gallai.
\newblock On directed paths and circuits.
\newblock In \emph{Theory of Graphs (Proc. Colloq. Titany, 1966)}, pages
  115--118. Academic Press, New York, 1968.

\bibitem{Gya92}
A. Gy\'arf\'as.
\newblock Graphs with $k$ odd cycle lengths.
\newblock {\it Discrete Math.}, 103, pp. 41--48, 1992.


\bibitem{Has64}
M.~Hasse.
\newblock Zur algebraischen bergr\"und der graphentheorie {I}.
\newblock \emph{Math. Nachr.}, 28: 275--290, 1964.

\bibitem{HoTa73}
J. Hopcroft and R. Tarjan.
\newblock Efficient algorithms for graph manipulation.
\newblock {\it Communications of the ACM}, 16 (6): 372--378, 1973.


\bibitem{KRS11}
T. Kaiser, O. Ruck\'y, and R. Skrekovski.
\newblock Graphs with odd cycle lengths 5 and 7 are 3-colorable.
\newblock {\it SIAM J. Discrete Math.},25(3):1069--1088, 2011.

\bibitem{KKPM}
R. Kim, SJ. Kim, J. Ma; B. Park
\newblock Cycles with two blocks in $k$-chromatic digraphs
\newblock arXiv:1610.05839

\bibitem{LRS10}
C. L\"owenstein, D. Rautenbach, and I. Schiermeyer.
\newblock Cycle length parities and the chromate number.
\newblock {\it J. Graph Theory}, 64(3):210--218, 2010.


\bibitem{MiSc04}
P. Mih\'ok and I. Schiermeyer.
\newblock Cycle lengths and chromatic number of graphs.
\newblock {\it Discrete Math.}, 286(1-2): 147--149, 2004.

\bibitem{Roy67}
B.~Roy.
\newblock Nombre chromatique et plus longs chemins d'un graphe.
\newblock \emph{Rev. Francaise Informat. Recherche Op\'erationnelle},
  1 (5): 129--132, 1967.


\bibitem{Sum81}
D.~P. Sumner.
\newblock Subtrees of a graph and the chromatic number.
\newblock In {\em The theory and applications of graphs (Kalamazoo, Mich.,
  1980)}, pages 557--576. Wiley, New York, 1981.

\bibitem{Vit62}
L.~M. Vitaver.
\newblock Determination of minimal coloring of vertices of a graph by means of
  boolean powers of the incidence matrix.
\newblock \emph{Doklady Akademii Nauk SSSR}, 147: 758--759, 1962.


\bibitem{Wan08}
S.S. Wang.
\newblock Structure and coloring of graphs with only small odd cycles.
\newblock {\it SIAM J.Discrete Math.}, 22:1040--1072, 2008.


\end{thebibliography}

\end{document}
